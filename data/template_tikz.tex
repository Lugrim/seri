\documentclass{standalone}
\usepackage{tikz}

\begin{document}

% These set the width of a day and the height of an hour.
\newcommand*\daywidth{6cm}
\newcommand*\hourheight{4em}

% The entry style will have two options:
% * the first option sets how many hours the entry will be (i.e. its height);
% * the second option sets how many overlapping entries there are (thus
%   determining the width).
\tikzset{entry/.style 2 args={
    xshift=(0.5334em+0.8pt)/2,
    draw,
    line width=0.8pt,
    font=\sffamily,
    rectangle,
    rounded corners,
    fill=blue!20,
    anchor=north west,
    inner sep=0.3333em,
    text width={\daywidth/#2-1.2em-1.6pt},
    minimum height=#1*\hourheight,
    align=center
}}

\tikzset{talk/.style 2 args={
		entry={#1}{#2},
		fill=red!40
	}
}

\tikzset{meal/.style 2 args={
		entry={#1}{#2},
		fill=green!40
	}
}

\tikzset{fun/.style 2 args={
		entry={#1}{#2},
		fill=blue!40
	}
}

\tikzset{transport/.style 2 args={
		entry={#1}{#2},
		fill=gray!20
	}
}

\begin{tikzpicture}[y=-\hourheight,x=\daywidth]
{{ CALENDAR }}
\end{tikzpicture}
\end{document}
